Версия 1.3.1, 02.03.2016
\vspace{\topsep}
\begin{enumerate}
	\item Добавлено \textit{FinRes fall notification} в \hyperref[portfolio_notifications]{\ref{portfolio_notifications}параметры уведомлений портфеля}
	\item Добавлен раздел \hyperref[robot_formulas_examples]{\ref{robot_formulas_examples}Примеры написания \textit{Ratio formula}}
\end{enumerate}

\vspace{5mm}

\noindentВерсия 1.3.2, 21.04.2016
%\vspace{\topsep}
\begin{enumerate}
	\item Добавлено \textit{Type price} в параметры портфеля
	\item Добавлены \textit{Sell volume OB}, \textit{Buy volume OB} в параметры инструментов портфеля
	\item Изменено описание \textit{Percent of quantity}, \textit{Market volume}
	\item Добавлен раздел, описывающий правила использования \textit{Ratio formula} \textit{(доступно НЕ во всех версиях программы)}
\end{enumerate}

\vspace{5mm}

\noindentВерсия 1.3.3, 24.06.2016
%\vspace{\topsep}
\begin{enumerate}
	\item Добавлено автоматическое хеджирование позиции (параметр \textit{Hedge}) в параметры портфеля
	\item В таблицу с раздвижками добавлена возможность расчета средних цен раздвижек на покупку и продажу для заданного портфеля
	\item Изменен смысл переменной $curpos$ в формулах для подвижки лимитов, теперь это НЕ позиция портфеля, а позиция по \textit{Is first} бумаге портфеля.
		Это очень важное исправление, оно сказывается только на портфелях, у которых \textit{Count} по \textit{Is first} бумаге не равен $1$.
	\item Изменена формула для переменной $k3$ в формулах для подвижки лимитов, было:
		
		$k3 = \left(\abs{Lim\_Sell_0 - Lim\_Buy_0} - TP - K \right) \times \frac{V}{curpos}$ ,
		
		стало:
		
		$k3 = \left(\abs{Lim\_Sell_0 - Lim\_Buy_0 - TP} - K \right) \times \frac{V}{curpos}$ .
		
		Данное изменение связано с существовавшей до сих пор сменой знака переменной $k3$ в случае когда $\abs{Lim\_Sell_0 - Lim\_Buy_0} < TP$ (хотя, при правильных настройках
		такой ситуации не должно быть).
	\item Добавлены уведомления о сильном изменении $Lim\_Sell$ и $Lim\_Buy$
\end{enumerate}

\vspace{5mm}

\noindentВерсия 1.3.4, 21.07.2016
%\vspace{\topsep}
\begin{enumerate}
	\item Добавлена возможность выбора кода клиента (\textit{Client code}) с которого выставлять и снимать заявки по заданной бумаге
	\item Изменена формула для переменной $k3$ в формулах для подвижки лимитов, формула приняла свой первоначальный вид:
		
		$k3 = \left(\abs{Lim\_Sell_0 - Lim\_Buy_0} - TP - K \right) \times \frac{V}{curpos}$ .
\end{enumerate}

\vspace{5mm}

\noindentВерсия 1.3.5, 10.09.2016
%\vspace{\topsep}
\begin{enumerate}
	\item Добавлен telegram бот и описание его подключения
	\item Добавлен параметр \textit{Close} в настройки расписания
	\item Добавлен параметр \textit{To market} в настройки расписания
	\item Добавлено \textit{Too much not hedged notification} в \hyperref[portfolio_notifications]{\ref{portfolio_notifications}параметры уведомлений портфеля}
	\item Переработано \hyperref[algo_features]{\ref{algo_features}поведение робота при срабатывании стоп-лосса и таймера}
	\item Добавлен раздел \hyperref[round_trip]{\ref{round_trip}\textit{Подсчет скорости транзакций}}
	\item Добавлена параметр \textit{Type trade}
	\item Добавлена режим торговли \textit{Option hedge} в параметр \textit{Type}
\end{enumerate}

\vspace{5mm}

\noindentВерсия 1.3.6, 29.11.2016
%\vspace{\topsep}
\begin{enumerate}
	\item Убран параметр \textit{Log level}
    \item Добавлен параметр \textit{Count type}
    \item Добавлен параметр \textit{Count formula}
\end{enumerate}

\vspace{5mm}

\noindentВерсия 1.3.7, 20.12.2016
%\vspace{\topsep}
\begin{enumerate}
	\item Добавлен неочевидный момент работы робота
	\item \textit{Decimals} бумаги портфеля теперь отвечает за число знаков после десятичной точки в сделках по данной бумаге в таблице раздвижек
	\item \textit{Decimals} не просто влияет на число отображаемых знаков после десятичной точки в таблице, но и округляет все редактируемые дробные числа строки таблицы до заданного числа знаков
	\item Добавлен \textit{To0} в расписание
\end{enumerate}

\vspace{5mm}

\noindentВерсия 1.3.8, 20.01.2017
%\vspace{\topsep}
\begin{enumerate}
	\item Добавлено ограничение на количество портфелей и финансовых инструментов
	\item Добавлен \textit{FUT move limits} в параметры инструментов портфеля
	\item Добавлен \textit{SPOT move limits} в параметры инструментов портфеля
\end{enumerate}

\vspace{5mm}

\noindentВерсия 1.4.0, 05.02.2017
%\vspace{\topsep}
\begin{enumerate}
	\item Параметр \textit{Ratio formula} заменен на два параметра: \textit{Ratio sell formula} и \textit{Ratio buy formula}
	\item Добавлены параметры \textit{Custom trade} и \textit{Trade formula}
	\item Добавлен параметр \textit{Overlay}
	\item Все формулы теперь пишутся на языке программирования C++ с соответствующим API
	\item Добавлен параметр \textit{MM} в параметры бумаг портфеля
	\item \textit{K}, \textit{K1}, \textit{K2} и \textit{Avg opened} добавлены в основную таблицу
	\item Параметр \textit{Wait hedge} заменен на \textit{Max not hedged}, если вы хотите получить поведение, аналогичное \textit{Wait hedge}, необходимо задать \textit{Max not hedged} равным $1$
	\item Добавлен раздел с описанием редактора формул
	\item \hyperref[move]{\ref{move}Добавлен раздел с описанием работы приказа ''переместить заявку''}
\end{enumerate}

\vspace{5mm}

\noindentВерсия 1.4.2, 05.04.2017
%\vspace{\topsep}
\begin{enumerate}
	\item Добавлен параметр \textit{In formulas}
	\item Добавлен параметр \textit{Simply first}
	\item Добавлена всплывающая подсказка для \textit{Avg. sell}, \textit{Avg. buy} с ценами бумаг
	\item Добавлен параметр \textit{Save/load trades} в настройки приложения, позволяющий сохранять/загружать таблицу сделок при закрытии/открытии приложения
	\item Добавлен экспорт таблиц в \textit{CSV} файл
	\item Ограничен размер таблиц лога и сделок до 100000 строк, при переполнении ''старые'' записи будут автоматически удаляться
	\item Добавлен столбец \textit{Color}
	\item Изменены возможные значения параметра \textit{Type price}
	\item Добавлен параметр бумаги портфеля \textit{Calc price OB}
	\item Добавлен параметр бумаги портфеля \textit{Trading price OB}
	\item Добавлен параметр бумаги портфеля \textit{Depth OB}
	\item Удален параметр бумаги портфеля \textit{Buy volume OB}
	\item Удален параметр бумаги портфеля \textit{Sell volume OB}
\end{enumerate}

\vspace{5mm}

\noindentВерсия 1.4.3, 01.06.2017
%\vspace{\topsep}
\begin{enumerate}
	\item \hyperref[orderbook]{\ref{orderbook}Добавлен раздел с описанием особенностей использования торговых стаканов инструментов}
\end{enumerate}

\vspace{5mm}

\noindentВерсия 1.4.4, 11.08.2017
%\vspace{\topsep}
\begin{enumerate}
	\item Добавлена торговля ''в файл'': добавлено значение \textit{Client code}, равное \textit{To file}
	\item Добавлен фильтр некоторых сообщений в логе, такие сообщения будут отображаться в логе $1$ раз в $10$ секунд и будут отмечены в конце сообщения символом \textit{xN}, где \textit{N} -- количество непоказанных сообщений
\end{enumerate}

\vspace{5mm}

\noindentВерсия 1.4.5, 13.09.2017
%\vspace{\topsep}
\begin{enumerate}
	\item Добавлен \textit{Type} \textit{Find fastest}
	\item Добавлен \textit{Client code} \textit{Use fastest/*}
\end{enumerate}

\vspace{5mm}

\noindentВерсия 1.4.8, 5.12.2017
%\vspace{\topsep}
\begin{enumerate}
	\item Добавлен параметр \textit{Virtual 0 pos}
	\item Добавлен параметр \textit{Timetable only stop} в окно настройки расписания
	\item \hyperref[keys]{\ref{keys}Добавлен менеджер ключей подключений к биржам}
\end{enumerate}

\vspace{5mm}

\noindentВерсия 1.4.9, 27.12.2017
%\vspace{\topsep}
\begin{enumerate}
	\item Добавлен параметр \textit{BitMEX level to0} в параметры бумаги портфеля, подробнее в документации
	\item Добавлен параметр \textit{BitMEX level close} в параметры бумаги портфеля, подробнее в документации
	\item Добавлен параметр \textit{BitMEX only maker} в параметры бумаги портфеля, подробнее в документации
\end{enumerate}

\vspace{5mm}

\noindentВерсия 1.4.10, 24.01.2018
%\vspace{\topsep}
\begin{enumerate}
	\item Для повышения безопасности соединения изменен формат файла сертификата в настройках подключения
\end{enumerate}

\vspace{5mm}

\noindentВерсия 1.4.11, 15.03.2018
%\vspace{\topsep}
\begin{enumerate}
	\item \hyperref[manageconns]{\ref{manageconns}Добавлена возможность добавления/удаления подключений к биржам криптовалют}
	\item Добавлен информационный канал в мессенджере Telegram \href{https://t.me/fkviking\_info}{https://t.me/fkviking\_info}
	\item Изменено поведение параметра \textit{Price check}
	\item Параметр \textit{BitMEX only maker} переименован в \textit{Only maker} и теперь работает для BitMEX и Bitfinex
\end{enumerate}

\vspace{5mm}

\noindentВерсия 1.4.12, 20.03.2018
%\vspace{\topsep}
\begin{enumerate}
	\item В менеджер ключей добавлена возможность экспорта ключа
\end{enumerate}

\vspace{5mm}

\noindentВерсия 1.5.0, 02.04.2018
%\vspace{\topsep}
\begin{enumerate}
	\item Добавлена возможность добавления подключения к Deribit
	\item \hyperref[errors]{\ref{errors}Обновлен раздел с наиболее распространенными ошибками, возникающими при работе программы, и способами их устранения (а именно торговые ошибки)}
\end{enumerate}

\vspace{5mm}

\noindentВерсия 1.5.1, 16.04.2018
%\vspace{\topsep}
\begin{enumerate}
	\item \textit{BitMEX level to0} переименован в \textit{Level to0} и теперь доступен еще и для Deribit
	\item \textit{BitMEX level close} переименован в \textit{Level close} и теперь доступен еще и для Deribit
	\item \textit{Only maker} теперь работает для Deribit
	\item Изменена совместная работа параметров \textit{Simply first} и \textit{Only maker}
\end{enumerate}

\vspace{5mm}

\noindentВерсия 1.5.2, 20.05.2018
%\vspace{\topsep}
\begin{enumerate}
	\item Добавлены параметры портфеля \textit{Extra formulas}, \textit{Extra field\#1} и \textit{Extra field\#2}
\end{enumerate}

\vspace{5mm}

\noindentВерсия 1.5.3, 20.06.2018
%\vspace{\topsep}
\begin{enumerate}
	\item \hyperref[positions]{\ref{positions}Добавлено отображение позиции с некоторых криптовалютных бирж}
\end{enumerate}

\vspace{5mm}

\noindentВерсия 1.5.4, 10.08.2018
%\vspace{\topsep}
\begin{enumerate}
	\item Добавлена возможность добавления подключения к Cryptofacilities
	\item \textit{Level to0} теперь доступен еще и для Cryptofacilities
	\item \textit{Level close} теперь доступен еще и для Cryptofacilities
	\item \textit{Only maker} теперь доступен еще и для Cryptofacilities
\end{enumerate}

\vspace{5mm}

\noindentВерсия 1.5.5, 29.08.2018
%\vspace{\topsep}
\begin{enumerate}
	\item Добавлена возможность добавления подключения к KuCoin
	\item \hyperref[problems]{\ref{problems}Обновлен раздел с возможными проблемами и их решениями}
\end{enumerate}

\vspace{5mm}

\noindentВерсия 1.5.6, 15.10.2018
%\vspace{\topsep}
\begin{enumerate}
	\item \textit{Client code} не может быть пустым для бумаг с \textit{Count} отличным от нуля
	\item В окно отображание позиции добавлено отображание баланса по нескольким валютам
	\item Добавлена возможность получения значений индексов с CSIndex
\end{enumerate}

\vspace{5mm}

\noindentВерсия 1.5.7, 25.10.2018
%\vspace{\topsep}
\begin{enumerate}
	\item Добавлена возможность добавления подключения к CEX.IO
	\item Добавлена возможность сохранять значения между вызовами формул в поле \textit{data} (которое является словарем) портфеля
\end{enumerate}

\vspace{5mm}

\noindentВерсия 1.5.8, 12.12.2018
%\vspace{\topsep}
\begin{enumerate}
	\item Удалены \textit{Find fastest} и \textit{Use fastest}
	\item Изменен принцип работы \textit{Round robin}
\end{enumerate}

\vspace{5mm}

\noindentВерсия 1.5.9, 25.02.2019
%\vspace{\topsep}
\begin{enumerate}
	\item Добавлена возможность добавления подключения к Huobi Russia
\end{enumerate}

